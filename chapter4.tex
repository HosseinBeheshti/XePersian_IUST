\chapter{نتایج شبیه‌سازی و پیشنهادات}
\label{simulation}
\newpage

\section{شبیه‌سازی}
در این قسمت کارایی الگوریتم پیشنهادی در این پایان‌نامه با الگوریتم‌های موجود جهت بازیابی سیگنال‌های دیکشنری تنک  
(الگوریتم‌های موجود در 
\cite{Baraniuk2017})
بررسی شده است. شبیه‌سازی با استفاده از نرم‌افزار
\lr{MATLAB}
و بسته‌ی 
\lr{CVX}
انجام شده است. کد‌های شبیه‌سازی به صورت آنلاین در دسترس است. (جهت دسترسی به کد‌ها 
\href{https://gitlab.com/HosseinBeheshti/AdaptiveBDS}{اینجا}
\LTRfootnote{https://gitlab.com/HosseinBeheshti/AdaptiveBDS}
کلیک کنید)

بردار‌ها و ماتریس‌ها جهت شبیه‌سازی به صورت زیر تولید شده است.
درایه‌‌های ماتریس اندازه‌گیری 
$ a_{i,j} $
به صورت مستقل از توزیع نرمال استاندارد انتخاب شده است. اندیس مولفه‌های غیر صفر بردار
$s$-تنک
$ \bm{x}\in \R^{N}  $
با فرض 
$ N=1000 $
از توزیع یکنواخت و مقدار دامنه‌ی هر درایه از توزیع نرمال استاندارد انتخاب شده است.

جهت تولید دیکشنری 
$ \bm{D}\in \R^{n\times N} $
به ازای
$ N=1000$
و
$n=50 $
، ابتدا یک ماتریس با ستون‌های مستقل و تصادفی از 
$ S^{n-1} $
ایجاد شده است. سپس قاب چگال دیکشنری با استفاده از پایه‌های متعامد و نرمالیزه‌ی فضای ستونی ماتریس، ساخته شده است.


الگوریتم پیشنهادی این پایان‌نامه با دو الگوریتم مبتنی بر برنامه‌ریزی محدب ارائه شده در 
\cite{Baraniuk2017}
مقایسه شده است. اولین الگوریتم، با حل یک مساله‌ی برنامه‌ریزی خطی محدب
(\lr{LP}\LTRfootnote{Linear programming optimization})
و دومی با حل یک مساله‌ی مخروطی مرتبه‌ی دوم
(\lr{CP}\LTRfootnote{Second-order cone programming})
پاسخ را پیدا می‌کند
\cite[بخش ~\lr{4.1}]{Baraniuk2017}،
\cite[بخش ~\lr{4.2}]{Baraniuk2017}.
هر دو الگوریتم بیان شده از مدل نمونه‌برداری
\eqref{eq:SignAT}
استفاده می‌کنند.
مولفه‌های آستانه‌های مورد نیاز
($ \tau_{i} $) 
از یک توزیع گوسی با میانگین صفر و واریانس 
$ \sigma^{2} $
تولید شده است. 
در الگوریتم پیشنهادی، مقدار تخمین بالادستی نُرم
$ \bm{f} $
برابر با
$ r= 2\norm{\bm{f}}_{2} $
در نظر گرفته شده است.  برای دو الگوریتم
\lr{LP}
و
\lr{CP}
مقدار
$ \sigma=r $
در نظر گرفته شده است.
خطای نرمالیزه‌ی بازیابی برای هر سه الگوریتم به صورت 
$ \|\bm{f}-\hat{\bm{f}}\|_{2} /\|\bm{f}\|_{2} $
تعریف شده است. در الگوریتم پیشنهادی تعداد بلوک‌ها برابر با 
$ L=10 $
در نظر گرفته شده است.


برای محاسبه‌ی خطا در هر نمونه الگوریتم ۱۰۰ بار اجرا شده است و خطای نمایش داده شده برابر با میانگین خطا در تمام اجرا‌ها است.

\begin{figure}[t]
	\centering
	\includestandalone[scale=0.6]{Images/ch4/SimFig1}
	%\input{AdaptiveBD}
	\caption{میانگین خطای بازیابی الگوریتم‌های \lr{LP}
	،
	\lr{CP}
	و الگوریتم پیشنهادی}
	\label{fig:SimFig1}
\end{figure}

شکل
\ref{fig:SimFig1}
تنایج شبیه‌سازی به ازای
$s=10$
را نشان می‌دهد.
همانگونه که از شکل مشخص است، الگوریتم‌های
\lr{LP}
و
\lr{CP}
به ازای تعداد نمونه‌های برابر رفتار مشابه دارند و خطای بازیابی این دو الگوریتم به صورت میانگین
\lr{$ 2 $dB}
با یکدیگر اختلاف دارد.
الگوریتم پیشنهادی ما، در تعداد نمونه‌های کم، مقداری نسبت به دو الگوریتم
\lr{LP}
و
\lr{CP}
از خطای بازیابی بیشتری برخوردار است ولی با افزایش تعداد نمونه‌ها به نظر می‌رسد که یک تغییر فاز در میزان خطای بازیابی رخ می‌دهد. همانگونه که از شکل
\ref{fig:SimFig1}
مشخص است، الگوریتم پیشنهادی ما، به میزان قابل توجهی در حالت پایدار بهتر عمل می‌کند. مقدار تفاوت در این شبیه‌سازی به صورت میانگین
\lr{$\bm{50} $\textbf{dB}}
است.


در شبیه‌سازی دوم، کارایی الگوریتم پیشنهادی به ازای مقادیر مختلف درجه‌ی تنکی بررسی شده است. شکل 
\ref{fig:SimFig2}
نتایج شبیه‌سازی به ازای 
$ s=10,20,30,40,50 $
را نشان می‌دهد. در این شبیه‌سازی تمامی پارامتر‌های دیگر مشابه شبیه‌سازی قبلی در نظر گرفته شده است. همانگونه که از شکل مشخص است با افزایش درجه‌ی تنکی میزان خطای بازیابی نیز افزایش می‌یابد.
\begin{figure}[t]
	\centering
	\includestandalone[scale=0.6]{Images/ch4/SimFig2}
	%\input{AdaptiveBD}
	\caption{رفتار الگوریتم ارائه شده به ازای مقادیر مختلف درجه تنکی}
	\label{fig:SimFig2}
\end{figure}

\newpage
\section{نتیجه‌گیری و پیشنهادات}
در این پایان‌نامه، الگوریتم وفقی جهت استفاده از اطلاعات قبلی سیگنال ارائه گردید. با استفاده از این روش، خطای بازیابی، به طرز قابل ملاحظه‌ای کاهش می‌یابد. علاوه بر این، کاهش خطای بازیابی در الگوریتم ارائه شده، به صورت نمایی است. در مقابل به ازای افزایش دقت بازیابی، پیچیدگی نمونه‌برداری افزایش می‌یابد که در برخی کاربردهای عملیاتی، استفاده از روش پیشنهادی را محدود می‌کند. جهت اثبات نتایج از تئوری هندسه‌ی ابعاد بالا استفاده گردیده است.
در این پایان‌نامه، ما از شهود هندسی جهت بیان رفتار الگوریتم استفاده نموده‌ایم. رویکرد بیان شده قابل استفاده در سایر حوزه‌های پردازش سیگنال است. 

در این طرح از الگوی ثابت جهت کاهش واریانس آستانه‌ها استفاده شده است. جهت هوشمند سازی فرآیند نمونه‌برداری، پیشنهاد می‌شود که در هر مرحله از الگوریتم، خصوصیات هندسی مجموعه‌ی شدنی (به عنوان مثال عرض میانگین) اندازه‌گیری گردد.  با استفاده از این خصوصیات می‌توان در حالاتی که تخمین‌گر به مقدار مطلوب از خطای بازیابی رسیده است، از نمونه‌برداری اضافه نیز جلوگیری نمود.





