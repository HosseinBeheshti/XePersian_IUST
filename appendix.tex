\chapter{پیوست‌ها}
در این بخش اثبات دو قضیه‌ی
\ref{theorem:thm8}
و 
\ref{thm.main}
آورده شده است. در ادامه ابتدا به اثبات قضیه‌ی
\ref{theorem:thm8}
می‌پردازیم و با استفاده از نتایج این قضیه، قضیه‌ی اصلی، با استفاده از استقراء اثبات شده است. 
\section{اثبات قضیه‌ی
\ref{theorem:thm8}}
\label{Appndx1}
جهت اثبات ابتدا چند لم مورد استفاده در روند اثبات بیان شده است. در ادامه با توجه به مدل سیگنال در نظر گرفته شده،‌ قضیه‌ی 
\ref{theorem:thm7}
به مجموعه‌ی سیگنال‌های دیکشنری تنک، گسترش یافته است.
\begin{lemma}
\cite[لم~8.25]{foucart2013mathematical}
\label{lmm:slepian}
فرض کنید 
$\bm{x}$
و
$\bm{y}$
بردار‌های تصادفی گوسی با میانگین صفر در
$\R^{m}$
باشند. در این صورت اگر
\begin{align}
\label{eq:slep1}
\mathbb{E}\vert x_{i}-x_{j} \vert^{2} \leq \mathbb{E}\vert y_{i}-y_{j} \vert^{2} \quad \text{for all}~ i,j\in [m]
\end{align}
باشد، رابطه‌ی زیر برقرار است.
\begin{align}
\label{eq:slep2}
\mathbb{E} \max_{j\in [m]} x_{j} \leq \mathbb{E} \max_{j\in [m]} y_{j} 
\end{align}
\end{lemma}

لم
\ref{lmm:slepian}
در مراجع
\lr{Slepian}
نامیده می‌شود.

\begin{lemma}
\label{lmm:SparsW}
\cite[لم~2.3]{plan2013robust}
(عرض میانگین مجموعه‌ی سیگنال‌های 
$s$-تنک)
عرض میانگین مجموعه‌ی
\begin{align}
\label{eq:SparsW1}
S_{n,s} := \lbrace \bm{x} \in \R^{n}~:~\norm{\bm{x}}_{0}\leq s,~\norm{\bm{x}}_{2}\leq 1\rbrace
\end{align}
در رابطه‌ی زیر صدق می‌کند.
\begin{align}
\label{eq:SparsW2}
cs \log\left(2n/s\right)\leq \mathit{w}^{2}\left(S_{n,s}\right)\leq Cs \log\left(2n/s\right)
\end{align}
\end{lemma}	

همانگونه که در متن اشاره گردید، جهت اثبات نتایج در حالت وفقی (با آستانه‌های غیر صفر) از نتایج حسگری فشرده‌ی تک بیتی سنتی (آستانه‌ی صفر) استفاده شده است. لم 
\ref{lmm:Dpluse}
بیانگر ارتباط بردار‌ها در بعد
$n$
و
$n+1$
است.
\begin{lemma}
\label{lmm:Dpluse}
برای دو بردار
$\tilde{\bm{f}},\tilde{\bm{g}}\in \R^{n+1}$
که به صورت 
\begin{align}
\label{eq:Dpluse1}
	\tilde{\bm{f}} :=
	\begin{bmatrix}
	\begin{array}{c}
	\bm{f}_{[n]} \\
	\hline
	f_{n+1}
	\end{array}
	\end{bmatrix}
	\quad
	\tilde{\bm{g}} :=
	\begin{bmatrix}
	\begin{array}{c}
	\bm{g}_{[n]} \\
	\hline
	g_{n+1}
	\end{array}
	\end{bmatrix}	
\end{align}
تشکیل شده است و در آن
$f_{n+1},g_{n+1}\neq 0$
است، داریم	
\begin{align}
\label{eq:Dpluse2}
\norm{\dfrac{\bm{f}_{[n]}}{f_{n+1}}-\dfrac{\bm{g}_{[n]}}{g_{n+1}}}_{2} \leq 	\dfrac{\|\tilde{\bm{f}}\|_{2}\|\tilde{\bm{g}}\|_{2}}{\vert f_{n+1}\vert \vert g_{n+1}\vert} \norm{\dfrac{\tilde{\bm{f}}}{\|\tilde{\bm{f}}\|_{2}}-\dfrac{\tilde{\bm{g}}}{\|\tilde{\bm{g}}\|_{2}}}_{2}.
\end{align}
\end{lemma}
\begin{proof}
جهت اثبات رابطه‌ی فوق از نامساوی مثلثی در
$\R^{n}$
و نامساوی کوشی-شوارتز در 
$\R^{2}$
استفاده شده است. 
\begin{align*}
\norm{\dfrac{\bm{f}_{[n]}}{f_{n+1}}-\dfrac{\bm{g}_{[n]}}{g_{n+1}}}_{2} =& \|\tilde{\bm{f}}\|_{2} \norm{\dfrac{1/f_{n+1}}{\|\tilde{\bm{f}}\|_{2}}\bm{f}_{[n]} - \dfrac{1/g_{n+1}}{\|\tilde{\bm{f}}\|_{2}}\bm{g}_{[n]}  }_{2} \\ 
\end{align*}
با استفاده از نامساوی مثلثی داریم
\begin{align*}
\norm{\dfrac{\bm{f}_{[n]}}{f_{n+1}}-\dfrac{\bm{g}_{[n]}}{g_{n+1}}}_{2} \leq & \|\tilde{\bm{f}}\|_{2} \left( \dfrac{1}{f_{n+1}} \norm{\dfrac{\tilde{\bm{f}}_{[n]}}{\|\tilde{\bm{f}}\|_{2}} -\dfrac{\tilde{\bm{g}}_{[n]}}{\|\tilde{\bm{g}}\|_{2}} }_{2} + \left| \dfrac{1/g_{n+1}}{\|\tilde{\bm{f}}\|_{2}}- \dfrac{1/f_{n+1}}{\|\tilde{\bm{g}}\|_{2}}\right| \norm{\bm{g}_{[n]}}_{2} \right) \\
=& \|\tilde{\bm{f}}\|_{2} \left( \dfrac{1}{f_{n+1}} \norm{\dfrac{\tilde{\bm{f}}_{[n]}}{\|\tilde{\bm{f}}\|_{2}} -\dfrac{\tilde{\bm{g}}_{[n]}}{\|\tilde{\bm{g}}\|_{2}} }_{2} + \dfrac{\norm{\bm{g}_{[n]}}_{2}}{\vert f_{n+1} \vert \vert g_{n+1} \vert} 
\left| \dfrac{f_{n+1}}{\|\tilde{\bm{f}}\|_{2}}- \dfrac{g_{n+1}}{\|\tilde{\bm{g}}\|_{2}}\right| \right) \\
\end{align*}
با بکار بردن نامساوی کوشی-شوارتز داریم	
\begin{align*}
\norm{\dfrac{\bm{f}_{[n]}}{f_{n+1}}-\dfrac{\bm{g}_{[n]}}{g_{n+1}}}_{2} \leq & \|\tilde{\bm{f}}\|_{2}  \left[ \dfrac{1}{\vert f_{n+1} \vert^{2}} + \dfrac{\norm{\bm{g}_{[n]}}_{2}^{2} }{ \vert f_{n+1} \vert^{2} \vert g_{n+1} \vert^{2}} \right]^{1/2} \\ 
&\left[ \norm{\dfrac{\bm{f}_{[n]}}{\|\tilde{\bm{f}}\|_{2}} -\dfrac{\bm{g}_{[n]}}{\|\tilde{\bm{g}}\|_{2}} }_{2}^{2} 
+ \left| \dfrac{f_{n+1}}{\|\tilde{\bm{f}}\|_{2}} -\dfrac{g_{n+1}}{\|\tilde{\bm{g}}\|_{2}} \right|^{2}
\right]^{1/2}\\
=& \|\tilde{\bm{f}}\|_{2} \left[ \dfrac{\norm{\tilde{\bm{g}}}_{2}^{2} }{ \vert f_{n+1} \vert^{2} \vert g_{n+1} \vert^{2}} \right]^{1/2} \norm{\dfrac{\tilde{\bm{f}}}{\|\tilde{\bm{f}}\|_{2}} - \dfrac{\tilde{\bm{g}}}{\norm{\tilde{\bm{g}}}_{2}}}_{2}
\end{align*}
با مرتب نمودن عبارت آخر  سمت راست معادله‌ی
\eqref{eq:Dpluse2}
به دست می‌آید.
\end{proof}


\begin{lemma}
\label{lmm:PreImW}
مجموعه‌ی سیگنال‌های 
$\ell_2$
نرمالیزه‌ی 
$s$-تنک
تحلیلی موثر در شرط زیر صدق می‌کنند.
\begin{align}
\label{eq:PreImW1}
\mathit{w}\left(\left(\mathbf{D}^{\ast}\right)^{-1}\left(\Sigma_{s}^{N,\text{eff}}\right)\cap S^{n-1} \right)\leq C \sqrt{s \log\left(N/s\right)}
\end{align}
\end{lemma}


\begin{proof}
با استفاده از تعریف عرض گوسی برای
$ \mathcal{K}_{s}:= \left(\mathbf{D}^{\ast}\right)^{-1}\left(\Sigma_{s}^{N,\text{eff}}\right)\cap S^{n-1} $
که در آن 
$ \bm{g}\in \R^{n} $
نشان‌دهنده‌ی  بردار تصادفی نرمال استاندارد است، داریم:
\begin{align*}
\mathit{w}\left(\mathcal{K}_{s}\right)= \mathbb{E}\left[\sup_{\substack{\bm{D}^{\ast}\bm{f}\in \Sigma_{s}^{N,\text{eff}}\\ \|\bm{f}\|_2=1}} \langle\bm{f},\bm{g}\rangle \right] 
=& \mathbb{E}\left[\sup_{\substack{\bm{D}^{\ast}\bm{f}\in \Sigma_{s}^{N,\text{eff}}\\ \|\bm{D}^{\ast}\bm{f}\|_2=1}} \langle\bm{D}\bm{D}^{\ast}\bm{f},\bm{g}\rangle \right] \\
\leq & \mathbb{E}\left[\sup_{\substack{\bm{x}\in \Sigma_{s}^{N,\text{eff}}\\ \|\bm{x}\|_2=1}} \langle\bm{D}\bm{x},\bm{g}\rangle \right] 
\end{align*}
با توجه به اینکه فرض نمودیم که 
$\bm{D}$
از چارچوب فشرده برخوردار باشد
\eqref{eq:eq27}، 
$ \|\bm{D}\|_{2\rightarrow 2} = 1 $
است. همچنین برای هر 
$ \bm{x},\bm{x}^{\prime}\in \Sigma^{N,\text{eff}}_{s} \cap S^{n-1} $
داریم:
\begin{align*}
\mathbb{E}\left[ \langle\bm{D}\bm{x},\bm{g}\rangle - \langle\bm{D}\bm{x}^{\prime},\bm{g}^{\prime}\rangle \right]^{2} =& \mathbb{E}\left[ \langle \bm{Dx},\bm{g} \rangle^{2} \right]+\mathbb{E}\left[ \langle \bm{D}\bm{x}^{\prime},\bm{g}^{\prime} \rangle^{2} \right] \\
=& \|\bm{D}\bm{x}\|_{2}^{2}+\|\bm{D}\bm{x}^{\prime}\|_{2}^{2} 
\leq \|\bm{x}\|^{2}_{2}+\|\bm{x}^{\prime}\|^{2}_{2}\\
=& \mathbb{E}\left[ \langle\bm{x},\bm{g}\rangle - \langle\bm{x}^{\prime},\bm{g}^{\prime}\rangle \right]^{2} 
\end{align*} 
در ادامه با اعمال لم
\eqref{lmm:slepian}
داریم:
\begin{align}
\label{eq:PreImW2}
w\left(\mathcal{K}_{s}\right) \leq  \mathbb{E}\left[\sup_{\substack{\bm{x}\in \Sigma_{s}^{N,\text{eff}}\\ \|\bm{x}\|_2=1}} \langle\bm{x},\bm{g}\rangle \right] =  w \left( \Sigma^{N,\text{eff}}_{s}\cap S^{n-1}\right)
\end{align}

مقدار عبارت فوق با استفاده از کران بالای محاسبه شده در لم
\eqref{lmm:SparsW}
به صورت زیر به دست می‌آید.
\begin{align}
 w \left( \Sigma^{N,\text{eff}}_{s}\cap S^{n-1}\right) \leq c \sqrt{s \log\left(N/s\right)}
\end{align}

\end{proof}

در قضیه‌ی بعدی تعداد نمونه‌های لازم جهت برش‌کاری یک مجموعه‌ی دیکشنری تنک محاسبه شده است.

\begin{theorem}[برشکاری مجموعه‌ی دیکشنری تنک]
\label{thm:D-TES}
فرض کنید 
$\epsilon>0$
، 
$m\geq C\epsilon^{-6} s \log\left(N/s\right)$
و 
$\bm{A}$
ماتریس نمونه‌برداری با درایه‌های تصادفی نرمال استاندارد باشد. آنگاه، با احتمال 
\begin{align}
1- \exp(-c\epsilon^{2}m),
\end{align}
 سطرهای 
$\bm{A}$
مجموعه‌ی
$\mathcal{K} = \left(\bm{D}^{\ast}\right)^{-1}\left( \Sigma_{s}^{N,\text{eff}} \right)\cap S^{n-1}$
را 
\lr{$\epsilon$-tessellate}
می‌کند. به عبارت دیگر
\begin{align*}
\label{eq:D-TES1}
\left[ \bm{f},\bm{g}\in \mathcal{K} ~:~	\text{sign}\langle\mathbf{a}_{i},\mathbf{f}\rangle = \text{sign}\langle\mathbf{a}_{i},\mathbf{g}\rangle ,~ i=1,\cdots ,m\right] 
 \Rightarrow \|\mathbf{f}-\mathbf{g}\|_{2}\leq \epsilon 
\end{align*}
\end{theorem}
\begin{proof}
با استفاده از نتایج 
\ref{theorem:thm7}
با فرض
$\mathcal{K} = \left(\bm{D}^{\ast}\right)^{-1}\left( \Sigma_{s}^{N,\text{eff}} \right)\cap S^{n-1}$
نتیجه به ازای 
\begin{align}
m\geq C \epsilon^{-6} w(\mathcal{K})^{2}
\end{align}
برقرار است.
با استفاده از لم
\ref{lmm:PreImW}
مقدار کران خطا به دست می‌آید.
\end{proof}

پس از بیان مقدمات لازم، می‌توان کران بیان شده برای بازیابی تک مرحله‌ای را اثبات نمود.

\begin{proof}[اثبات قضیه‌ی \ref{theorem:thm8}]
جهت اثبات، سیگنال‌های انتقال‌یافته را به صورت زیر تعریف می‌کنیم.

\begin{align}
\tilde{\bm{f}} :=
\begin{bmatrix}
\begin{array}{c}
\bm{f} \\
\hline
\sigma
\end{array}
\end{bmatrix}
\quad
\tilde{\bm{D}} :=
\begin{bmatrix}
\begin{array}{c|c}
\bm{D} & \bm{0}\\
\hline
\bm{0} & \bm{1}
\end{array}
\end{bmatrix}
\quad
\tilde{\bm{A}} :=
\begin{bmatrix}
\begin{array}{c|c}
\bm{A}& 
\begin{array}{c}
-\tau_{1}/\sigma \\ \vdots \\-\tau_{m}/\sigma
\end{array}
\end{array}
\end{bmatrix}
\end{align}
علاوه بر این،
$f_{\varDelta}$
پاسخ مساله‌ی بهینه‌سازی
\eqref{eq:SSR}
و نسخه‌ی انتقال‌یافته‌ی آن را به صورت
$ \tilde{\bm{g}}=\begin{bmatrix}
\begin{array}{c}
\bm{f}_{\varDelta} \\
\hline
\sigma
\end{array}
\end{bmatrix} $	
تعریف می‌کنیم. با توجه به 
\begin{align*}
\|\tilde{\bm{D}}^{\ast} \tilde{\bm{g}}\|_{1} = \norm{
\begin{bmatrix}
\begin{array}{c}
\tilde{\bm{D}}^{\ast}\bm{f}_{\varDelta} \\
\hline
\sigma
\end{array}
\end{bmatrix} 	
}_{1} =& \|\tilde{\bm{D}}^{\ast}\bm{f}_{\varDelta}\|_{1}+\sigma 
\leq \|\tilde{\bm{D}}^{\ast}\bm{f}\|_{1}+\sigma \leq \sqrt{s}r+\sigma \\
\leq & \sqrt{r^2+\sigma^2}\sqrt{s+1}
\end{align*}
و
$ \|\tilde{\bm{D}}^{\ast} \tilde{\bm{g}}\|_{2} = \|\tilde{\bm{g}} \|_{2}= \sqrt{\|\tilde{\bm{f}}_{\varDelta}\|_{2}^{2}+\sigma^2} \geq \sigma$
دو سیگنال
$ \tilde{\bm{f}},\tilde{\bm{g}} $
به ازای
$ s^{\prime}:= (r^2/\sigma^2 +1)(s+1) $، 
$s^{\prime}$-تنک
تحلیلی است.
اگر به تعداد
$ m \geq C {\epsilon^{\prime}}^{-6} s^{\prime} \log{(N/s^{\prime})}$
نمونه به صورت
\begin{align}
\text{sign}(\tilde{\bm{A}}\tilde{\bm{f}})=\text{sign}(\tilde{\bm{A}}\tilde{\bm{g}})=\bm{y} 
\end{align}
دریافت کنیم. با اعمال قضیه‌ی
\ref{thm:D-TES}
بر سیگنال‌های
$ \tilde{\bm{f}}/\|\tilde{\bm{f}}\|_{2} $ 
و
$ \tilde{\bm{g}}/\|\tilde{\bm{g}}\|_{2} $ 
با احتمال شکست حداکثر
$ \gamma \exp{(-cm{\epsilon^{\prime}}^{2})} \leq \gamma \exp{(-c^{\prime}m\epsilon^{2}r^2\sigma^2/(r^2+\sigma^2)^{2} )} $
داریم:
\begin{align}
\norm{\dfrac{\tilde{\bm{f}}}{\|\tilde{\bm{f}}\|_{2}}-\dfrac{\tilde{\bm{g}}}{\|\tilde{\bm{g}}\|_{2}}}\leq \epsilon^{\prime}
\end{align}
با استفاده از نتیجه‌ی لم
\ref{lmm:Dpluse}
رابطه‌ی زیر به دست می‌آید.
\begin{align}
\norm{\dfrac{\bm{f}}{\sigma}-\dfrac{\bm{f}_{\varDelta}}{\sigma}}_{2}\leq \dfrac{r^2+\sigma^{2}}{\sigma^{2}} \epsilon^{\prime}, \\
 \|\bm{f}-\bm{f}_{\varDelta}\|_{2}\leq r\epsilon
\end{align}
\end{proof}


\section{اثبات قضیه‌ی 
\ref{thm.main}}
\label{Appndx2}

\begin{proof}[اثبات قضیه‌ی \ref{thm.main}]
با فرض سیگنال
$ \bm{f}\in (\bm{D}^{\ast})^{-1} \Sigma_{s}^{N,\text{eff}} $ 
و با استفاده از استقراء ثابت می‌کنیم که رابطه‌ی
\begin{align}
\|\bm{f}-\bm{f}_{t}\|_{2}\leq  r 2^{1-t},
\end{align}
برای 
$t \in [L]$
برقرار است.
حکم برای 
$t=1$
برقرار است ( مرحله‌ی اول معادل با بازیابی تک مرحله‌ای است). فرض می‌کنیم که حکم به ازای
$t-1$
برقرار باشد و ثابت خواهیم نمود که به ازای 
$t$
نیز برقرار است. در مرحله‌ی
$t-1$
 داریم:

\begin{align}
\label{eq:fn1}
\|\bm{f}-\bm{f}_{t-1}\|_{2}\leq  r 2^{2-t},
\end{align}
و همچنین
\begin{align}
\label{eq:fn2}
\|\bm{D}^{\ast}(\bm{f}-\bm{f}_{t-1})\|_{1} \leq  \sqrt{s} \|\bm{f}-\bm{f}_{t-1}\|_{2} \leq \sqrt{s} r 2^{1-t}
\end{align}
سیگنالهای
$\bm{f}-\bm{f}_{t-1}$
و
$f_{\varDelta}$
در مجموعه‌ی شدنی مساله‌ی بهینه‌سازی بازیابی تک‌مرحله‌ای قرار دارد و در قیود مساله‌ی بهینه‌سازی صدق می‌کنند.
با در نظر گرفتن دو رابطه‌ی 
\eqref{eq:fn1}
و
\eqref{eq:fn2}
می‌توان نوشت:
\begin{align}
\label{eq:fn3}
\|\bm{f}-\bm{f}_{t}\|_{2} \leq \|\bm{f}-\bm{f}_{t-1}-f_{\varDelta}\|_{2} \leq \epsilon  r 2^{2-t}
\end{align}
در این حالت به ازای
\begin{align}
\label{eq:fn4}
\epsilon = 1/2
\end{align}
 با توجه به قضیه‌ی
\ref{theorem:thm8}
با اخذ
\begin{align}
m \geq C(r/\sigma+\sigma/r)^{6}(r^{2}/\sigma^{2}+1)(1/2)^{-6} s \log(N/s)
\end{align}
نمونه، با احتمال حداقل
\begin{align}
1-\gamma \exp((-c^{\prime} b (1/2)^{2} r^{2}\sigma^{2})/(r^{2}+\sigma^{2})^{2})
\end{align}
داریم
\begin{align}
\label{eq:eqL}
\| \bm{f}-\bm{f}_{t}\|_{2} \leq (1/2) r 2^{2-t}=  r 2^{1-t}
\end{align}
عبارت 
\eqref{eq:eqL}
برابر با حکم در مرحله‌ی
$t$ام
است. 
\end{proof}